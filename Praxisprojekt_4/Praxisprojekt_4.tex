% !TeX program = lualatex
% Für echtes Arial/Calibri: lualatex/xelatex + biber verwenden.
% Fallback mit pdflatex funktioniert (sans: TeX Gyre Heros).

\documentclass[11pt,a4paper]{article}

% ---------- Pakete: Sprache, Schrift, Mikrotypografie ----------
\usepackage[a4paper,margin=2cm]{geometry} % 2,00 cm Ränder
\usepackage[ngerman,english]{babel}       % Silbentrennung DE/EN
\usepackage{microtype}                    % Mikrotypografie (Silbentrennung/Zeilenumbrüche)
\usepackage[T1]{fontenc}
\usepackage[utf8]{inputenc}
\usepackage{iftex}
\ifPDFTeX
% Fallback: pdfLaTeX (Arial-ähnlich)
\usepackage[scale=0.95]{tgheros}   % TeX Gyre Heros ~ Helvetica/Arial
\renewcommand{\familydefault}{\sfdefault}
\else
% Echte Systemschriften mit Lua/XeLaTeX:
\usepackage{fontspec}
\setmainfont{Arial}[
Ligatures=TeX,
Scale=1.0
]
% Optionaler alternativer Sans-Font:
% \setmainfont{Calibri}[Ligatures=TeX,Scale=1.0]
\fi

% ---------- Zeilen & Absätze ----------
\usepackage{setspace}
\onehalfspacing                           % 1,5-zeilig
\setlength{\parindent}{0pt}               % kein Einzug
\setlength{\parskip}{6pt}                 % 6 pt Abstand nach Absatz

% ---------- Mathe, Grafik, Tabellen ----------
\usepackage{amsmath,amssymb}
\usepackage{graphicx}
\usepackage{booktabs}
\usepackage{siunitx}
\sisetup{detect-all}                      % Sans-Serif in siunitx
\usepackage{csquotes}                     % saubere Anführungszeichen / Blockzitate
\usepackage{mwe}                          % Beispielbilder (example-image*)

% ---------- Bild-/Tabellenunterschriften (10 pt) ----------
\usepackage{caption}
\captionsetup{font=footnotesize,labelfont=bf,labelsep=colon}
\addto\captionsngerman{\renewcommand{\figurename}{Abb.}}
\addto\captionsngerman{\renewcommand{\tablename}{Tab.}}
\newcommand{\source}[1]{\caption*{\footnotesize #1}} % Quellenangabe 10 pt

% ---------- Überschriften: Größen/Abstände/Nummerntiefe ----------
\usepackage{titlesec}
% H1: 16 pt, 12/12
\titleformat{\section}{\bfseries\fontsize{16pt}{18pt}\selectfont}{\thesection}{0.6em}{}
\titlespacing*{\section}{0pt}{12pt}{12pt}
% H2: 14 pt, 12/6
\titleformat{\subsection}{\bfseries\fontsize{14pt}{16pt}\selectfont}{\thesubsection}{0.6em}{}
\titlespacing*{\subsection}{0pt}{12pt}{6pt}
% H3: 11 pt, 12/6
\titleformat{\subsubsection}{\bfseries\fontsize{11pt}{13pt}\selectfont}{\thesubsubsection}{0.6em}{}
\titlespacing*{\subsubsection}{0pt}{12pt}{6pt}
\setcounter{secnumdepth}{3}
\setcounter{tocdepth}{3}

% Jede Section (Ebene 1) startet auf neuer Seite
\usepackage{etoolbox}
\pretocmd{\section}{\clearpage}{}{}

% ---------- Inhaltsverzeichnis: nur Ebene 1 fett ----------
\usepackage{tocloft}
\renewcommand{\cftsecfont}{\bfseries}
\renewcommand{\cftsecpagefont}{}

% ---------- Fußnoten explizit 10 pt ----------
\makeatletter
\renewcommand\footnotesize{\@setfontsize\footnotesize{10pt}{12pt}}
\makeatother

% ---------- Seitenzahlen: zentriert im Fuß ----------
\usepackage{fancyhdr}
\pagestyle{fancy}
\fancyhf{}
\cfoot{\thepage}

% ---------- Hyperlinks schwarz (keine Farben/Rahmen) ----------
\usepackage[hidelinks]{hyperref}

% ---------- APA7 Literatur (Biber) + deutscher Mapping ----------
\usepackage[
style=apa,
backend=biber,
sorting=nyt,
uniquename=init,
maxcitenames=2, % "et al." ab 3
maxbibnames=99,
doi=true,
url=true,
dateabbrev=false, % Vollständige Datumsangaben
eprint=false, % Unterdrücke eprint-Felder wenn DOI vorhanden
isbn=false, % ISBN normalerweise nicht in APA7
giveninits=true % Nur Initialen für Vornamen
]{biblatex}
\DeclareLanguageMapping{ngerman}{ngerman-apa}

% Zusätzliche APA7-Konfigurationen
\ExecuteBibliographyOptions{maxbibnames=999} % Alle Autoren im Literaturverzeichnis
\ExecuteBibliographyOptions{giveninits=true} % Nur Initialen
\ExecuteBibliographyOptions{uniquename=init} % Eindeutigkeit durch Initialen

% DOI-Formatierung anpassen
\DeclareFieldFormat{doi}{%
  \mkbibacro{DOI}\addcolon\space
  \ifhyperref
    {\href{https://doi.org/#1}{\nolinkurl{#1}}}
    {\nolinkurl{#1}}}

% Hängender Einzug 1.27 cm, 1,5-zeilig wie Text
\setlength{\bibhang}{1.27cm}
\defbibenvironment{bibliography}
{\list
    {\printtext[labelnumberwidth]{\printfield[labelnumberwidth]{labelnumber}}}
    {\setlength{\leftmargin}{\bibhang}
        \setlength{\itemindent}{-\bibhang}
        \setlength{\itemsep}{\baselineskip} % 1.5-Zeilenabstand wie Text
        \setlength{\parsep}{0pt}}
    \renewcommand*{\makelabel}[1]{##1\hss}}
{\endlist}
{\item}

% Beispiel-Bibliothek im Dokument (kannst du ersetzen)
\begin{filecontents*}{\jobname.bib}
    @book{Goodfellow2016,
        author    = {Goodfellow, Ian and Bengio, Yoshua and Courville, Aaron},
        year      = {2016},
        title     = {Deep Learning},
        publisher = {MIT Press},
        address   = {Cambridge, MA},
        isbn      = {978-0262035613}
    }
    @book{Bishop2006,
        author    = {Bishop, Christopher M.},
        year      = {2006},
        title     = {Pattern Recognition and Machine Learning},
        publisher = {Springer},
        address   = {New York, NY},
        doi       = {10.1007/978-0-387-45528-0}
    }
    @book{Hastie2009,
        author    = {Hastie, Trevor and Tibshirani, Robert and Friedman, Jerome},
        year      = {2009},
        title     = {The Elements of Statistical Learning},
        subtitle  = {Data Mining, Inference, and Prediction},
        edition   = {2},
        publisher = {Springer},
        address   = {New York, NY},
        doi       = {10.1007/978-0-387-84858-7}
    }
    @article{Kingma2015,
        author  = {Kingma, Diederik P. and Ba, Jimmy},
        year    = {2015},
        title   = {Adam: A Method for Stochastic Optimization},
        journaltitle = {Proceedings of the 3rd International Conference on Learning Representations},
        venue   = {San Diego, CA},
        eprint  = {1412.6980},
        eprinttype = {arxiv},
        url     = {https://arxiv.org/abs/1412.6980}
    }
    @incollection{Platt1999,
        author    = {Platt, John},
        year      = {1999},
        title     = {Probabilistic Outputs for Support Vector Machines and Comparisons to Regularized Likelihood Methods},
        booktitle = {Advances in Large Margin Classifiers},
        editor    = {Smola, Alexander J. and Bartlett, Peter and Schölkopf, Bernhard and Schuurmans, Dale},
        publisher = {MIT Press},
        address   = {Cambridge, MA},
        pages     = {61--74}
    }
    @online{Mueller2023,
        author = {Müller, Andreas and Schmidt, Maria},
        year   = {2023},
        title  = {Aktuelle Entwicklungen im Machine Learning},
        url    = {https://example.com/ml-trends},
        urldate = {2024-01-15}
    }
\end{filecontents*}
\addbibresource{\jobname.bib}

% ---------- Automatische Verzeichnisse nur bei ≥3 Einträgen ----------
\usepackage{totcount}
\regtotcounter{figure}
\regtotcounter{table}
\newcommand{\addtoTOC}[1]{\addcontentsline{toc}{section}{#1}}
\newcommand{\printlistsconditional}{%
    % Wirksam nach erneutem LaTeX-Lauf (Zähler aus .aux):
    \ifnum\totvalue{figure}>2
    \renewcommand{\listfigurename}{Abbildungsverzeichnis}
    \listoffigures
    \addtoTOC{Abbildungsverzeichnis}
    \clearpage
    \fi
    \ifnum\totvalue{table}>2
    \renewcommand{\listtablename}{Tabellenverzeichnis}
    \listoftables
    \addtoTOC{Tabellenverzeichnis}
    \clearpage
    \fi
}

% ---------- Blockzitat ≥ 40 Wörter (APA) ----------
\newenvironment{blockzitat}{%
    \begin{quote}\setlength{\leftskip}{1.27cm}\itshape\upshape\mdseries\selectfont
    }{\end{quote}}

% ---------- Meta-Felder für Titelseite ----------
\newcommand{\university}{IU Internationale Hochschule}
\newcommand{\studyprogram}{Studiengang (z.\,B. B.\,Sc.\ Data Science)}
\newcommand{\thesistype}{Art der Arbeit (z.\,B. Hausarbeit / Bachelorarbeit)}
\newcommand{\papertitle}{Titel der Arbeit}
\newcommand{\authorname}{Vorname Nachname}
\newcommand{\matno}{Matrikelnummer}
\newcommand{\address}{Straße Hausnr., PLZ Ort}
\newcommand{\advisor}{Betreuer:in, akademischer Titel}
\newcommand{\submissiondate}{TT.MM.JJJJ}

% ============================================================
%                         DOKUMENT
% ============================================================
\begin{document}
    \selectlanguage{ngerman}

    % ---------- Titelblatt (zählt als I, ohne Zahl) ----------
    \pagenumbering{Roman}
    \setcounter{page}{1}
    \begin{titlepage}
        \thispagestyle{empty}
        \begin{center}
            \large \university\\[6pt]
            \studyprogram\\[18pt]
            \textbf{\Large \thesistype}\\[24pt]
            {\bfseries\fontsize{18pt}{20pt}\selectfont \papertitle}\\[24pt]
        \end{center}
        \vspace{6mm}
        \begin{tabular}{@{}ll}
            Autor:            & \authorname \\
            Matrikelnummer:   & \matno \\
            Anschrift:        & \address \\
            Betreuung:        & \advisor \\
            Abgabedatum:      & \submissiondate \\
        \end{tabular}
        \vfill
        \begin{center}
            \footnotesize

        \end{center}
    \end{titlepage}

    % ---------- Erklärung / Sperrvermerk (optional je nach Arbeit) ----------
    \section*{Erklärung / Sperrvermerk}
    \addtoTOC{Erklärung / Sperrvermerk}
    Hier ggf. die Eigenständigkeits- und Sperrvermerkserklärung gemäß Vorgaben der Hochschule.
    \clearpage

    % ---------- Danksagung (optional) ----------
    \section*{Danksagung}
    \addtoTOC{Danksagung}
    Optionaler Text für Danksagungen.
    \clearpage

    % ---------- Abstracts (Deutsch & Englisch, je ca. 200 Wörter) ----------
    \section*{Abstract (Deutsch)}
    \addtoTOC{Abstract (Deutsch)}
    Kurzfassung der Arbeit (ca. 200 Wörter): Problemstellung, Methode, Ergebnisse, Implikationen.
    \clearpage

    \begin{otherlanguage*}{english}
        \section*{Abstract (English)}
        \addtoTOC{Abstract (English)}
        Abstract (approx. 200 words): problem, method, results, implications.
    \end{otherlanguage*}
    \clearpage

    % ---------- Inhaltsverzeichnis ----------
    \renewcommand{\contentsname}{Inhaltsverzeichnis}
    \tableofcontents
    \clearpage

    % ---------- Abbildungs-/Tabellenverzeichnis (nur bei ≥3) ----------
    \printlistsconditional

    % ---------- Abkürzungsverzeichnis ----------
    \section*{Abkürzungsverzeichnis}
    \addtoTOC{Abkürzungsverzeichnis}
    \begin{tabular}{@{}ll}
        \textbf{AUROC} & Area under the Receiver Operating Characteristic \\
        \textbf{BA}    & Balanced Accuracy \\
        \textbf{SNR}   & Signal-zu-Rausch-Verhältnis \\
    \end{tabular}
    \clearpage

    % ---------- Hauptteil: arabische Seitenzahlen ab "Einleitung" ----------
    \pagenumbering{arabic}
    \setcounter{page}{1}

    \section{Einleitung}
    Hier beginnt der Fließtext der Arbeit. Jede Section (Ebene 1) startet auf neuer Seite.
    Zitieren nach APA7, z.\,B. \parencite{Goodfellow2016,Bishop2006}. Direktzitat ≤40 Wörter „mit
    Anführungszeichen“ \parencite[S.~123]{Hastie2009}. Längeres Direktzitat:
    \begin{blockzitat}
        Dies ist ein Blockzitat (≥ 40 Wörter) ohne Anführungszeichen; links um \SI{1.27}{cm} eingerückt.
        Quellenangabe folgt nach APA: (Autor, Jahr, S.~xx).
    \end{blockzitat}

    \subsection{Zielsetzung und Aufbau}
    Kurze Darstellung der Ziele und der Gliederung.

    \subsubsection{Beitrag}
    Klarer Beitrag (Problem, Ansatz, Nutzen).

    \section{Theoretischer Hintergrund}
    Relevante Modelle/Begriffe, z.\,B. Regularisierung, Bias/Varianz. Formeln werden gesetzt:
    \begin{align}
        \hat{R}(\theta) &= \frac{1}{N}\sum_{i=1}^{N}\ell\!\left(y_i,f_\theta(x_i)\right) + \lambda\lVert\theta\rVert_2^2,\\
        \ell(y,\hat{y}) &= -\big[y\log\hat{y} + (1-y)\log(1-\hat{y})\big].
    \end{align}

    \section{Methodik}
    Design, Daten, Preprocessing, Metriken, Validierung.
    \subsection{Daten}
    Kurzbeschreibung der Datensätze.
    \subsection{Modelle und Hyperparameter}
    Tabellenbeispiel mit Quellenangabe (10 pt):
    \begin{table}[h]
        \centering
        \begin{tabular}{lcc}
            \toprule
            \textbf{Parameter} & \textbf{Wert A} & \textbf{Wert B} \\
            \midrule
            Lernrate & 0{,}001 & 0{,}01 \\
            Batchgröße & 64 & 64 \\
            \bottomrule
        \end{tabular}
        \caption{Beispielhafte Hyperparameter.}
        \source{Eigene Darstellung.}
        \label{tab:hyper}
    \end{table}

    \section{Ergebnisse}
    Beispielabbildung mit Titel und Quelle (10 pt):
    \begin{figure}[h]
        \centering
        \includegraphics[width=0.85\textwidth]{example-image}
        \caption{Schematische Lernkurven.}
        \source{Eigene Darstellung (Platzhalter).}
        \label{fig:learn}
    \end{figure}

    \section{Diskussion}
    Ergebnisse interpretieren, Limitationen, Implikationen.

    \section{Fazit}
    Zentrale Punkte, Ausblick, Handlungsempfehlungen.

    % ---------- Literaturverzeichnis ----------
    \clearpage
    \printbibliography[title={Literaturverzeichnis}]

    % ---------- Anhangsverzeichnis (bei Bedarf) ----------
    \clearpage
    \section*{Anhangsverzeichnis}
    \addtoTOC{Anhangsverzeichnis}
    \begin{itemize}
        \item Anhang A: Zusatzabbildungen
        \item Anhang B: Pseudocode
    \end{itemize}
    \clearpage

    % ---------- Anhänge ----------
    \appendix
    \section{Zusatzabbildungen}
    (Optionaler Inhalt des Anhangs.)

    \section{Pseudocode}
    (Optionaler Inhalt des Anhangs.)

        % ---------- Nützliche LaTeX-Referenz ----------

    \clearpage
    \section*{Nützliche LaTeX-Referenz}
    \addtoTOC{Nützliche LaTeX-Referenz}

    \subsection*{Zitieren nach APA 7 (biblatex-apa)}
    Indirektes Zitat: \verb|\parencite{Goodfellow2016}| → (Goodfellow et al., 2016)\\
    Mit Seitenzahl: \verb|\parencite[S.~123]{Bishop2006}|\\
    Direktzitat ≤40 Wörter: „…“ \verb|\parencite[S.~45]{Hastie2009}|\\
    Blockzitat ≥40 Wörter:
    \begin{verbatim}
        \begin{blockzitat}
            Langes Zitat ohne Anführungszeichen …
        \end{blockzitat}
    \end{verbatim}

    \subsection*{Abbildungen}
    \begin{verbatim}
        \begin{figure}[h]
            \centering
            \includegraphics[width=0.85\textwidth]{pfad/zur/datei}
            \caption{Titel der Abbildung.}
            \source{Quelle: Eigene Darstellung / Autor, Jahr, S.~xx.}
            \label{fig:beispiel}
        \end{figure}
    \end{verbatim}
    Querverweis: „siehe Abb.~\verb|\ref{fig:beispiel}|“.

    \subsection*{Tabellen}
    \begin{verbatim}
        \begin{table}[h]
            \centering
            \begin{tabular}{lcc}
                \toprule
                \textbf{Variable} & \textbf{Gruppe A} & \textbf{Gruppe B} \\
                \midrule
                x & 1{,}23 & 4{,}56 \\
                \bottomrule
            \end{tabular}
            \caption{Titel der Tabelle.}
            \source{Quelle: Eigene Darstellung.}
            \label{tab:beispiel}
        \end{table}
    \end{verbatim}
    Querverweis: „siehe Tab.~\verb|\ref{tab:beispiel}|“.

    \subsection*{Gleichungen}
    Einzeln:
    \begin{verbatim}
        \begin{equation}
            E = mc^2
        \end{equation}
    \end{verbatim}
    Mehrzeilig (nummeriert):
    \begin{verbatim}
        \begin{align}
            \hat{R}(\theta) &= \frac{1}{N}\sum_{i=1}^{N}\ell(y_i,f_\theta(x_i)) + \lambda\lVert\theta\rVert_2^2,\\
            \ell(y,\hat{y}) &= -\big[y\log\hat{y} + (1-y)\log(1-\hat{y})\big].
        \end{align}
    \end{verbatim}

    \subsection*{Listen}
    \begin{verbatim}
        \begin{itemize}
            \item Punkt A
            \item Punkt B
        \end{itemize}

        \begin{enumerate}
            \item Erstens
            \item Zweitens
        \end{enumerate}
    \end{verbatim}

    \subsection*{Fußnoten}
    \verb|Text\footnote{Inhalt der Fußnote in 10 pt.}|

    \subsection*{Einheiten und Zahlen (siunitx)}
    \verb|\SI{12,5}{\kilo\meter\per\hour}| → \SI{12,5}{\kilo\meter\per\hour}\\
    \verb|\num{12345,678}| → \num{12345,678}

    \subsection*{Quellen in Abbildungen/Tabellen}
    Direkt unter \verb|\caption| einfügen: \verb|\source{Quelle: …}| (10 pt).

    \subsection*{Platzhalter \& Blindtext}
    Platzhalterbild: \verb|\includegraphics{example-image}| (aus Paket \verb|mwe|).\\
    Kurzer Blindtext:
    \begin{verbatim}
        Lorem ipsum dolor sit amet, consectetur adipiscing elit.
    \end{verbatim}

    \subsection*{Bibliografie-Einträge (BibTeX mit Biber)}
    \textbf{Wichtige Eintragstypen:}
    \begin{verbatim}
@book{key,
  author = {Nachname, Vorname},
  year = {2023},
  title = {Titel des Buches},
  subtitle = {Untertitel (optional)},
  publisher = {Verlag},
  address = {Ort},
  edition = {2}, % nur bei 2. Auflage oder höher
  doi = {10.1000/xyz}
}

@article{key,
  author = {Nachname, Vorname and Zweiter, Autor},
  year = {2023},
  title = {Titel des Artikels},
  journaltitle = {Name der Zeitschrift},
  volume = {42},
  number = {3},
  pages = {123--145},
  doi = {10.1000/xyz}
}

@online{key,
  author = {Nachname, Vorname},
  year = {2023},
  title = {Titel der Webseite},
  url = {https://example.com},
  urldate = {2024-01-15}
}
    \end{verbatim}
    
    \textbf{Biber-spezifische Felder:}
    \begin{itemize}
        \item \verb|journaltitle| statt \verb|journal| (APA-konform)
        \item \verb|location| statt \verb|address| (moderne biblatex-Syntax)
        \item \verb|date| statt \verb|year| für komplexere Datumsangaben
    \end{itemize}

    \subsection*{Erweiterte LaTeX-Tipps}
    \textbf{Mathematik:}
    \begin{itemize}
        \item Inline-Mathe: \verb|$E = mc^2$| → $E = mc^2$
        \item Display-Mathe: \verb|\[E = mc^2\]| (unnummeriert)
        \item Nummerierte Gleichung: \verb|\begin{equation}...\end{equation}|
        \item Griechische Buchstaben: \verb|\alpha, \beta, \gamma| → $\alpha, \beta, \gamma$
    \end{itemize}
    
    \textbf{Querverweise:}
    \begin{itemize}
        \item Label setzen: \verb|\label{fig:beispiel}|
        \item Verweis: \verb|\ref{fig:beispiel}| oder \verb|\autoref{fig:beispiel}|
        \item Seitenverweis: \verb|\pageref{fig:beispiel}|
    \end{itemize}
    
    \textbf{Typografie:}
    \begin{itemize}
        \item Geschützte Leerzeichen: \verb|Abb.~\ref{fig:1}|
        \item Anführungszeichen: \verb|\enquote{Text}| (sprachabhängig)
        \item Gedankenstrich: \verb|--| (Bindestrich), \verb|---| (Gedankenstrich)
        \item Auslassungspunkte: \verb|\ldots| → …
    \end{itemize}
    
    \textbf{Häufige Probleme und Lösungen:}
    \begin{itemize}
        \item Biber-Cache löschen: \verb|biber --cache-clear|
        \item Umlaute: Verwende \verb|fontspec| mit LuaLaTeX/XeLaTeX
        \item Lange URLs: \verb|\url{...}| oder \verb|\href{url}{Text}|
        \item Overfull hbox: \verb|\sloppy| oder manuelle Zeilenumbrüche
    \end{itemize}

    \subsection*{Kompilierreihenfolge mit Biber}
    \textbf{Standard:} LuaLaTeX → Biber → LuaLaTeX → LuaLaTeX
    
    \textbf{VS Code/Automatisierung:}
    \begin{itemize}
        \item LaTeX Workshop Extension konfigurieren
        \item \verb|latexmkrc| für automatische Biber-Ausführung
        \item Overleaf nutzt automatisch die richtige Reihenfolge
    \end{itemize}


    % ========= Ende =========
\end{document}
